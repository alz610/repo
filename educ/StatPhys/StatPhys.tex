\documentclass[11pt]{article}


    
    \setcounter{secnumdepth}{0}

    \usepackage[breakable]{tcolorbox}
    \usepackage{parskip} % Stop auto-indenting (to mimic markdown behaviour)
    
    % Basic figure setup, for now with no caption control since it's done
    % automatically by Pandoc (which extracts ![](path) syntax from Markdown).
    \usepackage{graphicx}
    % Maintain compatibility with old templates. Remove in nbconvert 6.0
    \let\Oldincludegraphics\includegraphics
    % Ensure that by default, figures have no caption (until we provide a
    % proper Figure object with a Caption API and a way to capture that
    % in the conversion process - todo).
    \usepackage{caption}
    \DeclareCaptionFormat{nocaption}{}
    \captionsetup{format=nocaption,aboveskip=0pt,belowskip=0pt}

    \usepackage[Export]{adjustbox} % Used to constrain images to a maximum size
    \adjustboxset{max size={0.9\linewidth}{0.9\paperheight}}
    \usepackage{float}
    \floatplacement{figure}{H} % forces figures to be placed at the correct location
    \usepackage{xcolor} % Allow colors to be defined
    \usepackage{enumerate} % Needed for markdown enumerations to work
    \usepackage{geometry} % Used to adjust the document margins
    \usepackage{amsmath} % Equations
    \usepackage{amssymb} % Equations
    \usepackage{textcomp} % defines textquotesingle
    % Hack from http://tex.stackexchange.com/a/47451/13684:
    \AtBeginDocument{%
        \def\PYZsq{\textquotesingle}% Upright quotes in Pygmentized code
    }
    \usepackage{upquote} % Upright quotes for verbatim code
    \usepackage{eurosym} % defines \euro
    \usepackage[mathletters]{ucs} % Extended unicode (utf-8) support
    \usepackage{fancyvrb} % verbatim replacement that allows latex
    \usepackage{grffile} % extends the file name processing of package graphics 
                         % to support a larger range
    \makeatletter % fix for grffile with XeLaTeX
    \def\Gread@@xetex#1{%
      \IfFileExists{"\Gin@base".bb}%
      {\Gread@eps{\Gin@base.bb}}%
      {\Gread@@xetex@aux#1}%
    }
    \makeatother

    % The hyperref package gives us a pdf with properly built
    % internal navigation ('pdf bookmarks' for the table of contents,
    % internal cross-reference links, web links for URLs, etc.)
    \usepackage{hyperref}
    % The default LaTeX title has an obnoxious amount of whitespace. By default,
    % titling removes some of it. It also provides customization options.
    \usepackage{titling}
    \usepackage{longtable} % longtable support required by pandoc >1.10
    \usepackage{booktabs}  % table support for pandoc > 1.12.2
    \usepackage[inline]{enumitem} % IRkernel/repr support (it uses the enumerate* environment)
    \usepackage[normalem]{ulem} % ulem is needed to support strikethroughs (\sout)
                                % normalem makes italics be italics, not underlines
    \usepackage{mathrsfs}
     % load all other packages

    \usepackage{polyglossia}   %% загружает пакет многоязыковой вёрстки
    \setdefaultlanguage{russian}  %% устанавливает главный язык документа
        %\setdefaultlanguage[babelshorthands=true]{russian}  %% вместо предыдущей строки; доступны команды из пакета babel для русского языка
    \setotherlanguage{english} %% объявляет второй язык документа
    \defaultfontfeatures{Ligatures={TeX}}  %% свойства шрифтов по умолчанию. Для XeTeX опцию Renderer=Basic можно не указывать, она необходима для LuaTeX
    \setmainfont{CMU Serif} %% задаёт основной шрифт документа
    \setsansfont{CMU Sans Serif}                    %% задаёт шрифт без засечек
    \setmonofont{CMU Typewriter Text}               %% задаёт моноширинный шрифт
    
    \usepackage{empheq}

    \newcommand{\deriv}[1]{#1\hspace{0.25mm}'}


    
    % Colors for the hyperref package
    \definecolor{urlcolor}{rgb}{0,.145,.698}
    \definecolor{linkcolor}{rgb}{.71,0.21,0.01}
    \definecolor{citecolor}{rgb}{.12,.54,.11}

    % ANSI colors
    \definecolor{ansi-black}{HTML}{3E424D}
    \definecolor{ansi-black-intense}{HTML}{282C36}
    \definecolor{ansi-red}{HTML}{E75C58}
    \definecolor{ansi-red-intense}{HTML}{B22B31}
    \definecolor{ansi-green}{HTML}{00A250}
    \definecolor{ansi-green-intense}{HTML}{007427}
    \definecolor{ansi-yellow}{HTML}{DDB62B}
    \definecolor{ansi-yellow-intense}{HTML}{B27D12}
    \definecolor{ansi-blue}{HTML}{208FFB}
    \definecolor{ansi-blue-intense}{HTML}{0065CA}
    \definecolor{ansi-magenta}{HTML}{D160C4}
    \definecolor{ansi-magenta-intense}{HTML}{A03196}
    \definecolor{ansi-cyan}{HTML}{60C6C8}
    \definecolor{ansi-cyan-intense}{HTML}{258F8F}
    \definecolor{ansi-white}{HTML}{C5C1B4}
    \definecolor{ansi-white-intense}{HTML}{A1A6B2}
    \definecolor{ansi-default-inverse-fg}{HTML}{FFFFFF}
    \definecolor{ansi-default-inverse-bg}{HTML}{000000}

    % commands and environments needed by pandoc snippets
    % extracted from the output of `pandoc -s`
    \providecommand{\tightlist}{%
      \setlength{\itemsep}{0pt}\setlength{\parskip}{0pt}}
    \DefineVerbatimEnvironment{Highlighting}{Verbatim}{commandchars=\\\{\}}
    % Add ',fontsize=\small' for more characters per line
    \newenvironment{Shaded}{}{}
    \newcommand{\KeywordTok}[1]{\textcolor[rgb]{0.00,0.44,0.13}{\textbf{{#1}}}}
    \newcommand{\DataTypeTok}[1]{\textcolor[rgb]{0.56,0.13,0.00}{{#1}}}
    \newcommand{\DecValTok}[1]{\textcolor[rgb]{0.25,0.63,0.44}{{#1}}}
    \newcommand{\BaseNTok}[1]{\textcolor[rgb]{0.25,0.63,0.44}{{#1}}}
    \newcommand{\FloatTok}[1]{\textcolor[rgb]{0.25,0.63,0.44}{{#1}}}
    \newcommand{\CharTok}[1]{\textcolor[rgb]{0.25,0.44,0.63}{{#1}}}
    \newcommand{\StringTok}[1]{\textcolor[rgb]{0.25,0.44,0.63}{{#1}}}
    \newcommand{\CommentTok}[1]{\textcolor[rgb]{0.38,0.63,0.69}{\textit{{#1}}}}
    \newcommand{\OtherTok}[1]{\textcolor[rgb]{0.00,0.44,0.13}{{#1}}}
    \newcommand{\AlertTok}[1]{\textcolor[rgb]{1.00,0.00,0.00}{\textbf{{#1}}}}
    \newcommand{\FunctionTok}[1]{\textcolor[rgb]{0.02,0.16,0.49}{{#1}}}
    \newcommand{\RegionMarkerTok}[1]{{#1}}
    \newcommand{\ErrorTok}[1]{\textcolor[rgb]{1.00,0.00,0.00}{\textbf{{#1}}}}
    \newcommand{\NormalTok}[1]{{#1}}
    
    % Additional commands for more recent versions of Pandoc
    \newcommand{\ConstantTok}[1]{\textcolor[rgb]{0.53,0.00,0.00}{{#1}}}
    \newcommand{\SpecialCharTok}[1]{\textcolor[rgb]{0.25,0.44,0.63}{{#1}}}
    \newcommand{\VerbatimStringTok}[1]{\textcolor[rgb]{0.25,0.44,0.63}{{#1}}}
    \newcommand{\SpecialStringTok}[1]{\textcolor[rgb]{0.73,0.40,0.53}{{#1}}}
    \newcommand{\ImportTok}[1]{{#1}}
    \newcommand{\DocumentationTok}[1]{\textcolor[rgb]{0.73,0.13,0.13}{\textit{{#1}}}}
    \newcommand{\AnnotationTok}[1]{\textcolor[rgb]{0.38,0.63,0.69}{\textbf{\textit{{#1}}}}}
    \newcommand{\CommentVarTok}[1]{\textcolor[rgb]{0.38,0.63,0.69}{\textbf{\textit{{#1}}}}}
    \newcommand{\VariableTok}[1]{\textcolor[rgb]{0.10,0.09,0.49}{{#1}}}
    \newcommand{\ControlFlowTok}[1]{\textcolor[rgb]{0.00,0.44,0.13}{\textbf{{#1}}}}
    \newcommand{\OperatorTok}[1]{\textcolor[rgb]{0.40,0.40,0.40}{{#1}}}
    \newcommand{\BuiltInTok}[1]{{#1}}
    \newcommand{\ExtensionTok}[1]{{#1}}
    \newcommand{\PreprocessorTok}[1]{\textcolor[rgb]{0.74,0.48,0.00}{{#1}}}
    \newcommand{\AttributeTok}[1]{\textcolor[rgb]{0.49,0.56,0.16}{{#1}}}
    \newcommand{\InformationTok}[1]{\textcolor[rgb]{0.38,0.63,0.69}{\textbf{\textit{{#1}}}}}
    \newcommand{\WarningTok}[1]{\textcolor[rgb]{0.38,0.63,0.69}{\textbf{\textit{{#1}}}}}
    
    
    % Define a nice break command that doesn't care if a line doesn't already
    % exist.
    \def\br{\hspace*{\fill} \\* }
    % Math Jax compatibility definitions
    \def\gt{>}
    \def\lt{<}
    \let\Oldtex\TeX
    \let\Oldlatex\LaTeX
    \renewcommand{\TeX}{\textrm{\Oldtex}}
    \renewcommand{\LaTeX}{\textrm{\Oldlatex}}
    % Document parameters
    % Document title
    
    
    
    
% Pygments definitions
\makeatletter
\def\PY@reset{\let\PY@it=\relax \let\PY@bf=\relax%
    \let\PY@ul=\relax \let\PY@tc=\relax%
    \let\PY@bc=\relax \let\PY@ff=\relax}
\def\PY@tok#1{\csname PY@tok@#1\endcsname}
\def\PY@toks#1+{\ifx\relax#1\empty\else%
    \PY@tok{#1}\expandafter\PY@toks\fi}
\def\PY@do#1{\PY@bc{\PY@tc{\PY@ul{%
    \PY@it{\PY@bf{\PY@ff{#1}}}}}}}
\def\PY#1#2{\PY@reset\PY@toks#1+\relax+\PY@do{#2}}

\expandafter\def\csname PY@tok@w\endcsname{\def\PY@tc##1{\textcolor[rgb]{0.73,0.73,0.73}{##1}}}
\expandafter\def\csname PY@tok@c\endcsname{\let\PY@it=\textit\def\PY@tc##1{\textcolor[rgb]{0.25,0.50,0.50}{##1}}}
\expandafter\def\csname PY@tok@cp\endcsname{\def\PY@tc##1{\textcolor[rgb]{0.74,0.48,0.00}{##1}}}
\expandafter\def\csname PY@tok@k\endcsname{\let\PY@bf=\textbf\def\PY@tc##1{\textcolor[rgb]{0.00,0.50,0.00}{##1}}}
\expandafter\def\csname PY@tok@kp\endcsname{\def\PY@tc##1{\textcolor[rgb]{0.00,0.50,0.00}{##1}}}
\expandafter\def\csname PY@tok@kt\endcsname{\def\PY@tc##1{\textcolor[rgb]{0.69,0.00,0.25}{##1}}}
\expandafter\def\csname PY@tok@o\endcsname{\def\PY@tc##1{\textcolor[rgb]{0.40,0.40,0.40}{##1}}}
\expandafter\def\csname PY@tok@ow\endcsname{\let\PY@bf=\textbf\def\PY@tc##1{\textcolor[rgb]{0.67,0.13,1.00}{##1}}}
\expandafter\def\csname PY@tok@nb\endcsname{\def\PY@tc##1{\textcolor[rgb]{0.00,0.50,0.00}{##1}}}
\expandafter\def\csname PY@tok@nf\endcsname{\def\PY@tc##1{\textcolor[rgb]{0.00,0.00,1.00}{##1}}}
\expandafter\def\csname PY@tok@nc\endcsname{\let\PY@bf=\textbf\def\PY@tc##1{\textcolor[rgb]{0.00,0.00,1.00}{##1}}}
\expandafter\def\csname PY@tok@nn\endcsname{\let\PY@bf=\textbf\def\PY@tc##1{\textcolor[rgb]{0.00,0.00,1.00}{##1}}}
\expandafter\def\csname PY@tok@ne\endcsname{\let\PY@bf=\textbf\def\PY@tc##1{\textcolor[rgb]{0.82,0.25,0.23}{##1}}}
\expandafter\def\csname PY@tok@nv\endcsname{\def\PY@tc##1{\textcolor[rgb]{0.10,0.09,0.49}{##1}}}
\expandafter\def\csname PY@tok@no\endcsname{\def\PY@tc##1{\textcolor[rgb]{0.53,0.00,0.00}{##1}}}
\expandafter\def\csname PY@tok@nl\endcsname{\def\PY@tc##1{\textcolor[rgb]{0.63,0.63,0.00}{##1}}}
\expandafter\def\csname PY@tok@ni\endcsname{\let\PY@bf=\textbf\def\PY@tc##1{\textcolor[rgb]{0.60,0.60,0.60}{##1}}}
\expandafter\def\csname PY@tok@na\endcsname{\def\PY@tc##1{\textcolor[rgb]{0.49,0.56,0.16}{##1}}}
\expandafter\def\csname PY@tok@nt\endcsname{\let\PY@bf=\textbf\def\PY@tc##1{\textcolor[rgb]{0.00,0.50,0.00}{##1}}}
\expandafter\def\csname PY@tok@nd\endcsname{\def\PY@tc##1{\textcolor[rgb]{0.67,0.13,1.00}{##1}}}
\expandafter\def\csname PY@tok@s\endcsname{\def\PY@tc##1{\textcolor[rgb]{0.73,0.13,0.13}{##1}}}
\expandafter\def\csname PY@tok@sd\endcsname{\let\PY@it=\textit\def\PY@tc##1{\textcolor[rgb]{0.73,0.13,0.13}{##1}}}
\expandafter\def\csname PY@tok@si\endcsname{\let\PY@bf=\textbf\def\PY@tc##1{\textcolor[rgb]{0.73,0.40,0.53}{##1}}}
\expandafter\def\csname PY@tok@se\endcsname{\let\PY@bf=\textbf\def\PY@tc##1{\textcolor[rgb]{0.73,0.40,0.13}{##1}}}
\expandafter\def\csname PY@tok@sr\endcsname{\def\PY@tc##1{\textcolor[rgb]{0.73,0.40,0.53}{##1}}}
\expandafter\def\csname PY@tok@ss\endcsname{\def\PY@tc##1{\textcolor[rgb]{0.10,0.09,0.49}{##1}}}
\expandafter\def\csname PY@tok@sx\endcsname{\def\PY@tc##1{\textcolor[rgb]{0.00,0.50,0.00}{##1}}}
\expandafter\def\csname PY@tok@m\endcsname{\def\PY@tc##1{\textcolor[rgb]{0.40,0.40,0.40}{##1}}}
\expandafter\def\csname PY@tok@gh\endcsname{\let\PY@bf=\textbf\def\PY@tc##1{\textcolor[rgb]{0.00,0.00,0.50}{##1}}}
\expandafter\def\csname PY@tok@gu\endcsname{\let\PY@bf=\textbf\def\PY@tc##1{\textcolor[rgb]{0.50,0.00,0.50}{##1}}}
\expandafter\def\csname PY@tok@gd\endcsname{\def\PY@tc##1{\textcolor[rgb]{0.63,0.00,0.00}{##1}}}
\expandafter\def\csname PY@tok@gi\endcsname{\def\PY@tc##1{\textcolor[rgb]{0.00,0.63,0.00}{##1}}}
\expandafter\def\csname PY@tok@gr\endcsname{\def\PY@tc##1{\textcolor[rgb]{1.00,0.00,0.00}{##1}}}
\expandafter\def\csname PY@tok@ge\endcsname{\let\PY@it=\textit}
\expandafter\def\csname PY@tok@gs\endcsname{\let\PY@bf=\textbf}
\expandafter\def\csname PY@tok@gp\endcsname{\let\PY@bf=\textbf\def\PY@tc##1{\textcolor[rgb]{0.00,0.00,0.50}{##1}}}
\expandafter\def\csname PY@tok@go\endcsname{\def\PY@tc##1{\textcolor[rgb]{0.53,0.53,0.53}{##1}}}
\expandafter\def\csname PY@tok@gt\endcsname{\def\PY@tc##1{\textcolor[rgb]{0.00,0.27,0.87}{##1}}}
\expandafter\def\csname PY@tok@err\endcsname{\def\PY@bc##1{\setlength{\fboxsep}{0pt}\fcolorbox[rgb]{1.00,0.00,0.00}{1,1,1}{\strut ##1}}}
\expandafter\def\csname PY@tok@kc\endcsname{\let\PY@bf=\textbf\def\PY@tc##1{\textcolor[rgb]{0.00,0.50,0.00}{##1}}}
\expandafter\def\csname PY@tok@kd\endcsname{\let\PY@bf=\textbf\def\PY@tc##1{\textcolor[rgb]{0.00,0.50,0.00}{##1}}}
\expandafter\def\csname PY@tok@kn\endcsname{\let\PY@bf=\textbf\def\PY@tc##1{\textcolor[rgb]{0.00,0.50,0.00}{##1}}}
\expandafter\def\csname PY@tok@kr\endcsname{\let\PY@bf=\textbf\def\PY@tc##1{\textcolor[rgb]{0.00,0.50,0.00}{##1}}}
\expandafter\def\csname PY@tok@bp\endcsname{\def\PY@tc##1{\textcolor[rgb]{0.00,0.50,0.00}{##1}}}
\expandafter\def\csname PY@tok@fm\endcsname{\def\PY@tc##1{\textcolor[rgb]{0.00,0.00,1.00}{##1}}}
\expandafter\def\csname PY@tok@vc\endcsname{\def\PY@tc##1{\textcolor[rgb]{0.10,0.09,0.49}{##1}}}
\expandafter\def\csname PY@tok@vg\endcsname{\def\PY@tc##1{\textcolor[rgb]{0.10,0.09,0.49}{##1}}}
\expandafter\def\csname PY@tok@vi\endcsname{\def\PY@tc##1{\textcolor[rgb]{0.10,0.09,0.49}{##1}}}
\expandafter\def\csname PY@tok@vm\endcsname{\def\PY@tc##1{\textcolor[rgb]{0.10,0.09,0.49}{##1}}}
\expandafter\def\csname PY@tok@sa\endcsname{\def\PY@tc##1{\textcolor[rgb]{0.73,0.13,0.13}{##1}}}
\expandafter\def\csname PY@tok@sb\endcsname{\def\PY@tc##1{\textcolor[rgb]{0.73,0.13,0.13}{##1}}}
\expandafter\def\csname PY@tok@sc\endcsname{\def\PY@tc##1{\textcolor[rgb]{0.73,0.13,0.13}{##1}}}
\expandafter\def\csname PY@tok@dl\endcsname{\def\PY@tc##1{\textcolor[rgb]{0.73,0.13,0.13}{##1}}}
\expandafter\def\csname PY@tok@s2\endcsname{\def\PY@tc##1{\textcolor[rgb]{0.73,0.13,0.13}{##1}}}
\expandafter\def\csname PY@tok@sh\endcsname{\def\PY@tc##1{\textcolor[rgb]{0.73,0.13,0.13}{##1}}}
\expandafter\def\csname PY@tok@s1\endcsname{\def\PY@tc##1{\textcolor[rgb]{0.73,0.13,0.13}{##1}}}
\expandafter\def\csname PY@tok@mb\endcsname{\def\PY@tc##1{\textcolor[rgb]{0.40,0.40,0.40}{##1}}}
\expandafter\def\csname PY@tok@mf\endcsname{\def\PY@tc##1{\textcolor[rgb]{0.40,0.40,0.40}{##1}}}
\expandafter\def\csname PY@tok@mh\endcsname{\def\PY@tc##1{\textcolor[rgb]{0.40,0.40,0.40}{##1}}}
\expandafter\def\csname PY@tok@mi\endcsname{\def\PY@tc##1{\textcolor[rgb]{0.40,0.40,0.40}{##1}}}
\expandafter\def\csname PY@tok@il\endcsname{\def\PY@tc##1{\textcolor[rgb]{0.40,0.40,0.40}{##1}}}
\expandafter\def\csname PY@tok@mo\endcsname{\def\PY@tc##1{\textcolor[rgb]{0.40,0.40,0.40}{##1}}}
\expandafter\def\csname PY@tok@ch\endcsname{\let\PY@it=\textit\def\PY@tc##1{\textcolor[rgb]{0.25,0.50,0.50}{##1}}}
\expandafter\def\csname PY@tok@cm\endcsname{\let\PY@it=\textit\def\PY@tc##1{\textcolor[rgb]{0.25,0.50,0.50}{##1}}}
\expandafter\def\csname PY@tok@cpf\endcsname{\let\PY@it=\textit\def\PY@tc##1{\textcolor[rgb]{0.25,0.50,0.50}{##1}}}
\expandafter\def\csname PY@tok@c1\endcsname{\let\PY@it=\textit\def\PY@tc##1{\textcolor[rgb]{0.25,0.50,0.50}{##1}}}
\expandafter\def\csname PY@tok@cs\endcsname{\let\PY@it=\textit\def\PY@tc##1{\textcolor[rgb]{0.25,0.50,0.50}{##1}}}

\def\PYZbs{\char`\\}
\def\PYZus{\char`\_}
\def\PYZob{\char`\{}
\def\PYZcb{\char`\}}
\def\PYZca{\char`\^}
\def\PYZam{\char`\&}
\def\PYZlt{\char`\<}
\def\PYZgt{\char`\>}
\def\PYZsh{\char`\#}
\def\PYZpc{\char`\%}
\def\PYZdl{\char`\$}
\def\PYZhy{\char`\-}
\def\PYZsq{\char`\'}
\def\PYZdq{\char`\"}
\def\PYZti{\char`\~}
% for compatibility with earlier versions
\def\PYZat{@}
\def\PYZlb{[}
\def\PYZrb{]}
\makeatother


    % For linebreaks inside Verbatim environment from package fancyvrb. 
    \makeatletter
        \newbox\Wrappedcontinuationbox 
        \newbox\Wrappedvisiblespacebox 
        \newcommand*\Wrappedvisiblespace {\textcolor{red}{\textvisiblespace}} 
        \newcommand*\Wrappedcontinuationsymbol {\textcolor{red}{\llap{\tiny$\m@th\hookrightarrow$}}} 
        \newcommand*\Wrappedcontinuationindent {3ex } 
        \newcommand*\Wrappedafterbreak {\kern\Wrappedcontinuationindent\copy\Wrappedcontinuationbox} 
        % Take advantage of the already applied Pygments mark-up to insert 
        % potential linebreaks for TeX processing. 
        %        {, <, #, %, $, ' and ": go to next line. 
        %        _, }, ^, &, >, - and ~: stay at end of broken line. 
        % Use of \textquotesingle for straight quote. 
        \newcommand*\Wrappedbreaksatspecials {% 
            \def\PYGZus{\discretionary{\char`\_}{\Wrappedafterbreak}{\char`\_}}% 
            \def\PYGZob{\discretionary{}{\Wrappedafterbreak\char`\{}{\char`\{}}% 
            \def\PYGZcb{\discretionary{\char`\}}{\Wrappedafterbreak}{\char`\}}}% 
            \def\PYGZca{\discretionary{\char`\^}{\Wrappedafterbreak}{\char`\^}}% 
            \def\PYGZam{\discretionary{\char`\&}{\Wrappedafterbreak}{\char`\&}}% 
            \def\PYGZlt{\discretionary{}{\Wrappedafterbreak\char`\<}{\char`\<}}% 
            \def\PYGZgt{\discretionary{\char`\>}{\Wrappedafterbreak}{\char`\>}}% 
            \def\PYGZsh{\discretionary{}{\Wrappedafterbreak\char`\#}{\char`\#}}% 
            \def\PYGZpc{\discretionary{}{\Wrappedafterbreak\char`\%}{\char`\%}}% 
            \def\PYGZdl{\discretionary{}{\Wrappedafterbreak\char`\$}{\char`\$}}% 
            \def\PYGZhy{\discretionary{\char`\-}{\Wrappedafterbreak}{\char`\-}}% 
            \def\PYGZsq{\discretionary{}{\Wrappedafterbreak\textquotesingle}{\textquotesingle}}% 
            \def\PYGZdq{\discretionary{}{\Wrappedafterbreak\char`\"}{\char`\"}}% 
            \def\PYGZti{\discretionary{\char`\~}{\Wrappedafterbreak}{\char`\~}}% 
        } 
        % Some characters . , ; ? ! / are not pygmentized. 
        % This macro makes them "active" and they will insert potential linebreaks 
        \newcommand*\Wrappedbreaksatpunct {% 
            \lccode`\~`\.\lowercase{\def~}{\discretionary{\hbox{\char`\.}}{\Wrappedafterbreak}{\hbox{\char`\.}}}% 
            \lccode`\~`\,\lowercase{\def~}{\discretionary{\hbox{\char`\,}}{\Wrappedafterbreak}{\hbox{\char`\,}}}% 
            \lccode`\~`\;\lowercase{\def~}{\discretionary{\hbox{\char`\;}}{\Wrappedafterbreak}{\hbox{\char`\;}}}% 
            \lccode`\~`\:\lowercase{\def~}{\discretionary{\hbox{\char`\:}}{\Wrappedafterbreak}{\hbox{\char`\:}}}% 
            \lccode`\~`\?\lowercase{\def~}{\discretionary{\hbox{\char`\?}}{\Wrappedafterbreak}{\hbox{\char`\?}}}% 
            \lccode`\~`\!\lowercase{\def~}{\discretionary{\hbox{\char`\!}}{\Wrappedafterbreak}{\hbox{\char`\!}}}% 
            \lccode`\~`\/\lowercase{\def~}{\discretionary{\hbox{\char`\/}}{\Wrappedafterbreak}{\hbox{\char`\/}}}% 
            \catcode`\.\active
            \catcode`\,\active 
            \catcode`\;\active
            \catcode`\:\active
            \catcode`\?\active
            \catcode`\!\active
            \catcode`\/\active 
            \lccode`\~`\~ 	
        }
    \makeatother

    \let\OriginalVerbatim=\Verbatim
    \makeatletter
    \renewcommand{\Verbatim}[1][1]{%
        %\parskip\z@skip
        \sbox\Wrappedcontinuationbox {\Wrappedcontinuationsymbol}%
        \sbox\Wrappedvisiblespacebox {\FV@SetupFont\Wrappedvisiblespace}%
        \def\FancyVerbFormatLine ##1{\hsize\linewidth
            \vtop{\raggedright\hyphenpenalty\z@\exhyphenpenalty\z@
                \doublehyphendemerits\z@\finalhyphendemerits\z@
                \strut ##1\strut}%
        }%
        % If the linebreak is at a space, the latter will be displayed as visible
        % space at end of first line, and a continuation symbol starts next line.
        % Stretch/shrink are however usually zero for typewriter font.
        \def\FV@Space {%
            \nobreak\hskip\z@ plus\fontdimen3\font minus\fontdimen4\font
            \discretionary{\copy\Wrappedvisiblespacebox}{\Wrappedafterbreak}
            {\kern\fontdimen2\font}%
        }%
        
        % Allow breaks at special characters using \PYG... macros.
        \Wrappedbreaksatspecials
        % Breaks at punctuation characters . , ; ? ! and / need catcode=\active 	
        \OriginalVerbatim[#1,codes*=\Wrappedbreaksatpunct]%
    }
    \makeatother

    % Exact colors from NB
    \definecolor{incolor}{HTML}{303F9F}
    \definecolor{outcolor}{HTML}{D84315}
    \definecolor{cellborder}{HTML}{CFCFCF}
    \definecolor{cellbackground}{HTML}{F7F7F7}
    
    % prompt
    \makeatletter
    \newcommand{\boxspacing}{\kern\kvtcb@left@rule\kern\kvtcb@boxsep}
    \makeatother
    \newcommand{\prompt}[4]{
        \llap{{{\ttfamily\color{#2}[#3]:\hspace{3pt}}#4}}\vspace{-\baselineskip}
    }


    
    % Prevent overflowing lines due to hard-to-break entities
    \sloppy 
    % Setup hyperref package
    \hypersetup{
      breaklinks=true,  % so long urls are correctly broken across lines
      colorlinks=true,
      urlcolor=urlcolor,
      linkcolor=linkcolor,
      citecolor=citecolor,
      }
    % Slightly bigger margins than the latex defaults
    
    \geometry{verbose,tmargin=1in,bmargin=1in,lmargin=1in,rmargin=1in}
    
    

\begin{document}
    
    \title{StatPhys}

\date{\today}
\maketitle


    
    

    
    Выражение кинетической энергий вращения через угловую скорость:
\[E_\text{rot}(\vec\omega) = \frac 1 2 \left(I_1 \omega_1^2 + I_2 \omega_2^2 + I_3 \omega_3^2 \right)\]

    Выражение плотности вероятности угловой скорости:
\[\rho(\vec\omega) = C \, e^{-{E_\text{rot}(\vec\omega) \over T}}\] где
единица измерения \(T\) есть \(\text{erg}\)

    Определение момента импульса: \[\vec L = \hat I \cdot \vec \omega\]

Выражение тензора инерции в cистеме координат главных осей инерции:
\[\hat I = \begin{pmatrix} I_1 & 0 & 0 \\ 0 & I_2 & 0 \\ 0 & 0 & I_3 \end{pmatrix}\]
где \(I_i\) -- главный момент инерции

    Среднее значение непрерывной случайной величины \(X\):
\[\langle X \rangle = \int_{(X)} \! x \rho_X(x) \, \mathrm dx\]

Среднее значение преобразования случайной величины \(X\):
\[\langle g(X) \rangle = \int_{(X)} \! g(x) \, \rho_X(x) \, \mathrm dx\]

Условие нормировки: \[\int_{(X)} \! \rho_X(x) \, \mathrm dx = 1\]

где \(\rho_X\) и \((X)\) -- плотность вероятности и область определения
величины \(X\)

    \hypertarget{ux438ux43dux438ux446ux438ux430ux43bux438ux437ux430ux446ux438ux44f}{%
\section{Инициализация}\label{ux438ux43dux438ux446ux438ux430ux43bux438ux437ux430ux446ux438ux44f}}

    \begin{tcolorbox}[breakable, size=fbox, boxrule=1pt, pad at break*=1mm,colback=cellbackground, colframe=cellborder]
\prompt{In}{incolor}{156}{\boxspacing}
\begin{Verbatim}[commandchars=\\\{\}]
\PY{k+kn}{from} \PY{n+nn}{sympy} \PY{k+kn}{import} \PY{o}{*}
\PY{k+kn}{from} \PY{n+nn}{IPython}\PY{n+nn}{.}\PY{n+nn}{display} \PY{k+kn}{import} \PY{n}{Markdown}\PY{p}{,} \PY{n}{display}
\PY{k+kn}{import} \PY{n+nn}{numpy} \PY{k}{as} \PY{n+nn}{np}

\PY{n}{var}\PY{p}{(}\PY{l+s+s1}{\PYZsq{}}\PY{l+s+s1}{omega1 omega2 omega3 L1 L2 L3 C}\PY{l+s+s1}{\PYZsq{}}\PY{p}{,} \PY{n}{real}\PY{o}{=}\PY{k+kc}{True}\PY{p}{)}
\PY{n}{var}\PY{p}{(}\PY{l+s+s1}{\PYZsq{}}\PY{l+s+s1}{I1 I2 I3 T}\PY{l+s+s1}{\PYZsq{}}\PY{p}{,} \PY{n}{positive}\PY{o}{=}\PY{k+kc}{True}\PY{p}{)}

\PY{n}{E\PYZus{}rot\PYZus{}expr} \PY{o}{=} \PY{n}{S}\PY{p}{(}\PY{l+m+mi}{1}\PY{p}{)} \PY{o}{/} \PY{l+m+mi}{2} \PY{o}{*} \PY{p}{(}\PY{n}{I1} \PY{o}{*} \PY{n}{omega1}\PY{o}{*}\PY{o}{*}\PY{l+m+mi}{2} \PY{o}{+} \PY{n}{I2} \PY{o}{*} \PY{n}{omega2}\PY{o}{*}\PY{o}{*}\PY{l+m+mi}{2} \PY{o}{+} \PY{n}{I3} \PY{o}{*} \PY{n}{omega3}\PY{o}{*}\PY{o}{*}\PY{l+m+mi}{2}\PY{p}{)}
\end{Verbatim}
\end{tcolorbox}

    \hypertarget{ux437ux430ux434ux430ux447ux430-1}{%
\section{Задача 1}\label{ux437ux430ux434ux430ux447ux430-1}}

    \begin{tcolorbox}[breakable, size=fbox, boxrule=1pt, pad at break*=1mm,colback=cellbackground, colframe=cellborder]
\prompt{In}{incolor}{157}{\boxspacing}
\begin{Verbatim}[commandchars=\\\{\}]
\PY{n}{rho\PYZus{}expr} \PY{o}{=} \PY{n}{C} \PY{o}{*} \PY{n}{exp}\PY{p}{(}\PY{o}{\PYZhy{}} \PY{n}{E\PYZus{}rot\PYZus{}expr} \PY{o}{/} \PY{n}{T}\PY{p}{)}

\PY{n}{int\PYZus{}expr} \PY{o}{=} \PY{n}{integrate}\PY{p}{(}\PY{n}{rho\PYZus{}expr}\PY{p}{,}
                     \PY{p}{(}\PY{n}{omega1}\PY{p}{,} \PY{l+m+mi}{0}\PY{p}{,} \PY{n}{oo}\PY{p}{)}\PY{p}{,}
                     \PY{p}{(}\PY{n}{omega2}\PY{p}{,} \PY{l+m+mi}{0}\PY{p}{,} \PY{n}{oo}\PY{p}{)}\PY{p}{,}
                     \PY{p}{(}\PY{n}{omega3}\PY{p}{,} \PY{l+m+mi}{0}\PY{p}{,} \PY{n}{oo}\PY{p}{)}\PY{p}{)}

\PY{n}{C\PYZus{}set} \PY{o}{=} \PY{n}{solveset}\PY{p}{(}\PY{n}{int\PYZus{}expr} \PY{o}{\PYZhy{}} \PY{l+m+mi}{1}\PY{p}{,} \PY{n}{C}\PY{p}{)}

\PY{n}{Markdown}\PY{p}{(}\PY{l+s+sa}{fr}\PY{l+s+s1}{\PYZsq{}}\PY{l+s+s1}{\PYZdl{}}\PY{l+s+s1}{\PYZbs{}}\PY{l+s+s1}{displaystyle C }\PY{l+s+s1}{\PYZbs{}}\PY{l+s+s1}{in }\PY{l+s+si}{\PYZob{}}\PY{n}{latex}\PY{p}{(}\PY{n}{C\PYZus{}set}\PY{p}{)}\PY{l+s+si}{\PYZcb{}}\PY{l+s+s1}{\PYZdl{}}\PY{l+s+s1}{\PYZsq{}}\PY{p}{)}
\end{Verbatim}
\end{tcolorbox}
 
            
\prompt{Out}{outcolor}{157}{}
    
    \(\displaystyle C \in \left\{\frac{2 \sqrt{2} \sqrt{I_{1}} \sqrt{I_{2}} \sqrt{I_{3}}}{\pi^{\frac{3}{2}} T^{\frac{3}{2}}}\right\}\)

    

    \begin{tcolorbox}[breakable, size=fbox, boxrule=1pt, pad at break*=1mm,colback=cellbackground, colframe=cellborder]
\prompt{In}{incolor}{158}{\boxspacing}
\begin{Verbatim}[commandchars=\\\{\}]
\PY{n}{C\PYZus{}expr}\PY{p}{,} \PY{o}{=} \PY{n}{C\PYZus{}set}
\PY{n}{rho\PYZus{}expr\PYZus{}vec\PYZus{}omega} \PY{o}{=} \PY{n}{rho\PYZus{}expr}\PY{o}{.}\PY{n}{subs}\PY{p}{(}\PY{n}{C}\PY{p}{,} \PY{n}{C\PYZus{}expr}\PY{p}{)}

\PY{n}{Markdown}\PY{p}{(}\PY{l+s+sa}{fr}\PY{l+s+s1}{\PYZsq{}}\PY{l+s+s1}{\PYZdl{}}\PY{l+s+s1}{\PYZbs{}}\PY{l+s+s1}{displaystyle }\PY{l+s+s1}{\PYZbs{}}\PY{l+s+s1}{rho(}\PY{l+s+s1}{\PYZbs{}}\PY{l+s+s1}{vec}\PY{l+s+s1}{\PYZbs{}}\PY{l+s+s1}{omega) = }\PY{l+s+si}{\PYZob{}}\PY{n}{latex}\PY{p}{(}\PY{n}{rho\PYZus{}expr\PYZus{}vec\PYZus{}omega}\PY{p}{)}\PY{l+s+si}{\PYZcb{}}\PY{l+s+s1}{\PYZdl{}}\PY{l+s+s1}{\PYZsq{}}\PY{p}{)}
\end{Verbatim}
\end{tcolorbox}
 
            
\prompt{Out}{outcolor}{158}{}
    
    \(\displaystyle \rho(\vec\omega) = \frac{2 \sqrt{2} \sqrt{I_{1}} \sqrt{I_{2}} \sqrt{I_{3}} e^{\frac{- \frac{I_{1} \omega_{1}^{2}}{2} - \frac{I_{2} \omega_{2}^{2}}{2} - \frac{I_{3} \omega_{3}^{2}}{2}}{T}}}{\pi^{\frac{3}{2}} T^{\frac{3}{2}}}\)

    

    \hypertarget{ux437ux430ux434ux430ux447ux430-2}{%
\section{Задача 2}\label{ux437ux430ux434ux430ux447ux430-2}}

    Из определения момента импульса и выражения тензора инерции:
\[L_i = I_i \omega_i\]

    Выражение \(E_\text{rot}\) через \(L_i\):

    \begin{tcolorbox}[breakable, size=fbox, boxrule=1pt, pad at break*=1mm,colback=cellbackground, colframe=cellborder]
\prompt{In}{incolor}{159}{\boxspacing}
\begin{Verbatim}[commandchars=\\\{\}]
\PY{n}{E\PYZus{}rot\PYZus{}expr1} \PY{o}{=} \PY{n}{E\PYZus{}rot\PYZus{}expr}\PY{o}{.}\PY{n}{subs}\PY{p}{(}\PY{p}{[}\PY{p}{(}\PY{n}{omega1}\PY{p}{,} \PY{n}{L1} \PY{o}{/} \PY{n}{I1}\PY{p}{)}\PY{p}{,}
                               \PY{p}{(}\PY{n}{omega2}\PY{p}{,} \PY{n}{L2} \PY{o}{/} \PY{n}{I2}\PY{p}{)}\PY{p}{,}
                               \PY{p}{(}\PY{n}{omega3}\PY{p}{,} \PY{n}{L3} \PY{o}{/} \PY{n}{I3}\PY{p}{)}\PY{p}{]}\PY{p}{)}

\PY{n}{E\PYZus{}rot\PYZus{}expr1}
\end{Verbatim}
\end{tcolorbox}
 
            
\prompt{Out}{outcolor}{159}{}
    
    $\displaystyle \frac{L_{3}^{2}}{2 I_{3}} + \frac{L_{2}^{2}}{2 I_{2}} + \frac{L_{1}^{2}}{2 I_{1}}$

    

    \begin{tcolorbox}[breakable, size=fbox, boxrule=1pt, pad at break*=1mm,colback=cellbackground, colframe=cellborder]
\prompt{In}{incolor}{160}{\boxspacing}
\begin{Verbatim}[commandchars=\\\{\}]
\PY{n}{rho\PYZus{}expr} \PY{o}{=} \PY{n}{C} \PY{o}{*} \PY{n}{exp}\PY{p}{(}\PY{o}{\PYZhy{}} \PY{n}{E\PYZus{}rot\PYZus{}expr1} \PY{o}{/} \PY{n}{T}\PY{p}{)}

\PY{n}{int\PYZus{}expr} \PY{o}{=} \PY{n}{integrate}\PY{p}{(}\PY{n}{rho\PYZus{}expr}\PY{p}{,}
                     \PY{p}{(}\PY{n}{L1}\PY{p}{,} \PY{l+m+mi}{0}\PY{p}{,} \PY{n}{oo}\PY{p}{)}\PY{p}{,}
                     \PY{p}{(}\PY{n}{L2}\PY{p}{,} \PY{l+m+mi}{0}\PY{p}{,} \PY{n}{oo}\PY{p}{)}\PY{p}{,}
                     \PY{p}{(}\PY{n}{L3}\PY{p}{,} \PY{l+m+mi}{0}\PY{p}{,} \PY{n}{oo}\PY{p}{)}\PY{p}{)}

\PY{n}{C\PYZus{}set} \PY{o}{=} \PY{n}{solveset}\PY{p}{(}\PY{n}{int\PYZus{}expr} \PY{o}{\PYZhy{}} \PY{l+m+mi}{1}\PY{p}{,} \PY{n}{C}\PY{p}{)}

\PY{n}{Markdown}\PY{p}{(}\PY{l+s+sa}{fr}\PY{l+s+s1}{\PYZsq{}}\PY{l+s+s1}{\PYZdl{}}\PY{l+s+s1}{\PYZbs{}}\PY{l+s+s1}{displaystyle C }\PY{l+s+s1}{\PYZbs{}}\PY{l+s+s1}{in }\PY{l+s+si}{\PYZob{}}\PY{n}{latex}\PY{p}{(}\PY{n}{C\PYZus{}set}\PY{p}{)}\PY{l+s+si}{\PYZcb{}}\PY{l+s+s1}{\PYZdl{}}\PY{l+s+s1}{\PYZsq{}}\PY{p}{)}
\end{Verbatim}
\end{tcolorbox}
 
            
\prompt{Out}{outcolor}{160}{}
    
    \(\displaystyle C \in \left\{\frac{2 \sqrt{2}}{\pi^{\frac{3}{2}} \sqrt{I_{1}} \sqrt{I_{2}} \sqrt{I_{3}} T^{\frac{3}{2}}}\right\}\)

    

    \begin{tcolorbox}[breakable, size=fbox, boxrule=1pt, pad at break*=1mm,colback=cellbackground, colframe=cellborder]
\prompt{In}{incolor}{161}{\boxspacing}
\begin{Verbatim}[commandchars=\\\{\}]
\PY{n}{C\PYZus{}expr}\PY{p}{,} \PY{o}{=} \PY{n}{C\PYZus{}set}
\PY{n}{rho\PYZus{}expr\PYZus{}vec\PYZus{}L} \PY{o}{=} \PY{n}{rho\PYZus{}expr}\PY{o}{.}\PY{n}{subs}\PY{p}{(}\PY{n}{C}\PY{p}{,} \PY{n}{C\PYZus{}expr}\PY{p}{)}

\PY{n}{Markdown}\PY{p}{(}\PY{l+s+sa}{fr}\PY{l+s+s1}{\PYZsq{}}\PY{l+s+s1}{\PYZdl{}}\PY{l+s+s1}{\PYZbs{}}\PY{l+s+s1}{displaystyle }\PY{l+s+s1}{\PYZbs{}}\PY{l+s+s1}{rho(}\PY{l+s+s1}{\PYZbs{}}\PY{l+s+s1}{vec L) = }\PY{l+s+si}{\PYZob{}}\PY{n}{latex}\PY{p}{(}\PY{n}{rho\PYZus{}expr\PYZus{}vec\PYZus{}L}\PY{p}{)}\PY{l+s+si}{\PYZcb{}}\PY{l+s+s1}{\PYZdl{}}\PY{l+s+s1}{\PYZsq{}}\PY{p}{)}
\end{Verbatim}
\end{tcolorbox}
 
            
\prompt{Out}{outcolor}{161}{}
    
    \(\displaystyle \rho(\vec L) = \frac{2 \sqrt{2} e^{\frac{- \frac{L_{3}^{2}}{2 I_{3}} - \frac{L_{2}^{2}}{2 I_{2}} - \frac{L_{1}^{2}}{2 I_{1}}}{T}}}{\pi^{\frac{3}{2}} \sqrt{I_{1}} \sqrt{I_{2}} \sqrt{I_{3}} T^{\frac{3}{2}}}\)

    

    \hypertarget{ux437ux430ux434ux430ux447ux430-3}{%
\section{Задача 3}\label{ux437ux430ux434ux430ux447ux430-3}}

    Плотность вероятности компоненты угловой скорости \(\omega_i\):
\[\rho(\omega_i) = \int \limits_0^\infty \!\!\! \int \limits_0^\infty \!
\rho(\vec\omega) \, \mathrm d\omega_j  \mathrm d\omega_k \quad i \neq j \neq k\]

Среднее значение величины \(\omega_i\):
\[\langle \omega_i \rangle = \int \limits_0^\infty \! \omega_i \rho(\omega_i) \, \mathrm d\omega_i\]

Среднее значение преобразования \(g(\omega_i)\):
\[\langle g(\omega_i) \rangle = \int \limits_0^\infty \! g(\omega_i) \, \rho(\omega_i) \, \mathrm d\omega_i\]

    \begin{tcolorbox}[breakable, size=fbox, boxrule=1pt, pad at break*=1mm,colback=cellbackground, colframe=cellborder]
\prompt{In}{incolor}{193}{\boxspacing}
\begin{Verbatim}[commandchars=\\\{\}]
\PY{n}{omega} \PY{o}{=} \PY{n}{np}\PY{o}{.}\PY{n}{array}\PY{p}{(}\PY{p}{[}\PY{n}{omega1}\PY{p}{,} \PY{n}{omega2}\PY{p}{,} \PY{n}{omega3}\PY{p}{]}\PY{p}{)}
\PY{n}{n} \PY{o}{=} \PY{n+nb}{len}\PY{p}{(}\PY{n}{omega}\PY{p}{)}

\PY{k}{for} \PY{n}{i} \PY{o+ow}{in} \PY{n+nb}{range}\PY{p}{(}\PY{n}{n}\PY{p}{)}\PY{p}{:}
    \PY{n}{mask} \PY{o}{=} \PY{n}{np}\PY{o}{.}\PY{n}{ones}\PY{p}{(}\PY{n}{n}\PY{p}{,} \PY{n}{dtype}\PY{o}{=}\PY{n+nb}{bool}\PY{p}{)}
    \PY{n}{mask}\PY{p}{[}\PY{n}{i}\PY{p}{]} \PY{o}{=} \PY{k+kc}{False}

    \PY{n}{rho\PYZus{}expr\PYZus{}omega} \PY{o}{=} \PY{n}{integrate}\PY{p}{(}\PY{n}{rho\PYZus{}expr\PYZus{}vec\PYZus{}omega}\PY{p}{,}
                               \PY{o}{*}\PY{p}{[}\PY{p}{(}\PY{n}{omega\PYZus{}}\PY{p}{,} \PY{l+m+mi}{0}\PY{p}{,} \PY{n}{oo}\PY{p}{)} \PY{k}{for} \PY{n}{omega\PYZus{}} \PY{o+ow}{in} \PY{n}{omega}\PY{p}{[}\PY{n}{mask}\PY{p}{]}\PY{p}{]}\PY{p}{)}
    
    \PY{n}{overage\PYZus{}omega} \PY{o}{=} \PY{n}{integrate}\PY{p}{(}\PY{n}{omega}\PY{p}{[}\PY{n}{i}\PY{p}{]} \PY{o}{*} \PY{n}{rho\PYZus{}expr\PYZus{}omega}\PY{p}{,} \PY{p}{(}\PY{n}{omega}\PY{p}{[}\PY{n}{i}\PY{p}{]}\PY{p}{,} \PY{l+m+mi}{0}\PY{p}{,} \PY{n}{oo}\PY{p}{)}\PY{p}{)}
    \PY{n}{overage\PYZus{}omega\PYZus{}2} \PY{o}{=} \PY{n}{integrate}\PY{p}{(}\PY{n}{omega}\PY{p}{[}\PY{n}{i}\PY{p}{]}\PY{o}{*}\PY{o}{*}\PY{l+m+mi}{2} \PY{o}{*} \PY{n}{rho\PYZus{}expr\PYZus{}omega}\PY{p}{,} \PY{p}{(}\PY{n}{omega}\PY{p}{[}\PY{n}{i}\PY{p}{]}\PY{p}{,} \PY{l+m+mi}{0}\PY{p}{,} \PY{n}{oo}\PY{p}{)}\PY{p}{)}
    
    \PY{n}{expr} \PY{o}{=} \PY{n}{factor}\PY{p}{(}\PY{n}{overage\PYZus{}omega}\PY{o}{*}\PY{o}{*}\PY{l+m+mi}{2} \PY{o}{\PYZhy{}} \PY{n}{overage\PYZus{}omega\PYZus{}2}\PY{p}{)}

    \PY{n}{display}\PY{p}{(}\PY{n}{Markdown}\PY{p}{(}\PY{l+s+sa}{fr}\PY{l+s+s1}{\PYZsq{}}\PY{l+s+s1}{\PYZdl{}}\PY{l+s+s1}{\PYZbs{}}\PY{l+s+s1}{displaystyle }\PY{l+s+se}{\PYZbs{}}
\PY{l+s+s1}{        }\PY{l+s+s1}{\PYZbs{}}\PY{l+s+s1}{langle }\PY{l+s+s1}{\PYZbs{}}\PY{l+s+s1}{omega\PYZus{}}\PY{l+s+si}{\PYZob{}}\PY{n}{i}\PY{o}{+}\PY{l+m+mi}{1}\PY{l+s+si}{\PYZcb{}}\PY{l+s+s1}{ }\PY{l+s+s1}{\PYZbs{}}\PY{l+s+s1}{rangle\PYZca{}2 \PYZhy{} }\PY{l+s+s1}{\PYZbs{}}\PY{l+s+s1}{langle }\PY{l+s+s1}{\PYZbs{}}\PY{l+s+s1}{omega\PYZus{}}\PY{l+s+si}{\PYZob{}}\PY{n}{i}\PY{o}{+}\PY{l+m+mi}{1}\PY{l+s+si}{\PYZcb{}}\PY{l+s+s1}{\PYZca{}2 }\PY{l+s+s1}{\PYZbs{}}\PY{l+s+s1}{rangle = }\PY{l+s+si}{\PYZob{}}\PY{n}{latex}\PY{p}{(}\PY{n}{expr}\PY{p}{)}\PY{l+s+si}{\PYZcb{}}\PY{l+s+s1}{\PYZdl{}}\PY{l+s+s1}{\PYZsq{}}\PY{p}{)}\PY{p}{)}
\end{Verbatim}
\end{tcolorbox}

    \(\displaystyle \
 \langle \omega_1 \rangle^2 - \langle \omega_1^2 \rangle = - \frac{T \left(-2 + \pi\right)}{\pi I_{1}}\)

    
    \(\displaystyle \
 \langle \omega_2 \rangle^2 - \langle \omega_2^2 \rangle = - \frac{T \left(-2 + \pi\right)}{\pi I_{2}}\)

    
    \(\displaystyle \
 \langle \omega_3 \rangle^2 - \langle \omega_3^2 \rangle = - \frac{T \left(-2 + \pi\right)}{\pi I_{3}}\)

    
    \hypertarget{ux437ux430ux434ux430ux447ux430-4}{%
\section{Задача 4}\label{ux437ux430ux434ux430ux447ux430-4}}

    Плотность вероятности компоненты момента импульса \(L_i\):
\[\rho(L_i) = \int \limits_0^\infty \!\!\! \int \limits_0^\infty \!
\rho(\vec L) \, \mathrm dL_j \mathrm dL_k \quad i \neq j \neq k\]

Среднее значение величины \(L_i\):
\[\langle L_i \rangle = \int \limits_0^\infty \! L_i \rho(L_i) \, \mathrm dL_i\]

Среднее значение преобразования \(g(L_i)\):
\[\langle g(L_i) \rangle = \int \limits_0^\infty \! g(L_i) \, \rho(L_i) \, \mathrm dL_i\]

    \begin{tcolorbox}[breakable, size=fbox, boxrule=1pt, pad at break*=1mm,colback=cellbackground, colframe=cellborder]
\prompt{In}{incolor}{195}{\boxspacing}
\begin{Verbatim}[commandchars=\\\{\}]
\PY{n}{L} \PY{o}{=} \PY{n}{np}\PY{o}{.}\PY{n}{array}\PY{p}{(}\PY{p}{[}\PY{n}{L1}\PY{p}{,} \PY{n}{L2}\PY{p}{,} \PY{n}{L3}\PY{p}{]}\PY{p}{)}
\PY{n}{n} \PY{o}{=} \PY{n+nb}{len}\PY{p}{(}\PY{n}{L}\PY{p}{)}

\PY{k}{for} \PY{n}{i} \PY{o+ow}{in} \PY{n+nb}{range}\PY{p}{(}\PY{n}{n}\PY{p}{)}\PY{p}{:}
    \PY{n}{mask} \PY{o}{=} \PY{n}{np}\PY{o}{.}\PY{n}{ones}\PY{p}{(}\PY{n}{n}\PY{p}{,} \PY{n}{dtype}\PY{o}{=}\PY{n+nb}{bool}\PY{p}{)}
    \PY{n}{mask}\PY{p}{[}\PY{n}{i}\PY{p}{]} \PY{o}{=} \PY{k+kc}{False}

    \PY{n}{rho\PYZus{}expr\PYZus{}L} \PY{o}{=} \PY{n}{integrate}\PY{p}{(}\PY{n}{rho\PYZus{}expr\PYZus{}vec\PYZus{}L}\PY{p}{,}
                           \PY{o}{*}\PY{p}{[}\PY{p}{(}\PY{n}{L\PYZus{}}\PY{p}{,} \PY{l+m+mi}{0}\PY{p}{,} \PY{n}{oo}\PY{p}{)} \PY{k}{for} \PY{n}{L\PYZus{}} \PY{o+ow}{in} \PY{n}{L}\PY{p}{[}\PY{n}{mask}\PY{p}{]}\PY{p}{]}\PY{p}{)}
    
    \PY{n}{overage\PYZus{}L} \PY{o}{=} \PY{n}{integrate}\PY{p}{(}\PY{n}{L}\PY{p}{[}\PY{n}{i}\PY{p}{]} \PY{o}{*} \PY{n}{rho\PYZus{}expr\PYZus{}L}\PY{p}{,} \PY{p}{(}\PY{n}{L}\PY{p}{[}\PY{n}{i}\PY{p}{]}\PY{p}{,} \PY{l+m+mi}{0}\PY{p}{,} \PY{n}{oo}\PY{p}{)}\PY{p}{)}
    \PY{n}{overage\PYZus{}L\PYZus{}2} \PY{o}{=} \PY{n}{integrate}\PY{p}{(}\PY{n}{L}\PY{p}{[}\PY{n}{i}\PY{p}{]}\PY{o}{*}\PY{o}{*}\PY{l+m+mi}{2} \PY{o}{*} \PY{n}{rho\PYZus{}expr\PYZus{}L}\PY{p}{,} \PY{p}{(}\PY{n}{L}\PY{p}{[}\PY{n}{i}\PY{p}{]}\PY{p}{,} \PY{l+m+mi}{0}\PY{p}{,} \PY{n}{oo}\PY{p}{)}\PY{p}{)}
    
    \PY{n}{expr} \PY{o}{=} \PY{n}{factor}\PY{p}{(}\PY{n}{overage\PYZus{}L}\PY{o}{*}\PY{o}{*}\PY{l+m+mi}{2} \PY{o}{\PYZhy{}} \PY{n}{overage\PYZus{}L\PYZus{}2}\PY{p}{)}

    \PY{n}{display}\PY{p}{(}\PY{n}{Markdown}\PY{p}{(}\PY{l+s+sa}{fr}\PY{l+s+s1}{\PYZsq{}}\PY{l+s+s1}{\PYZdl{}}\PY{l+s+s1}{\PYZbs{}}\PY{l+s+s1}{displaystyle }\PY{l+s+se}{\PYZbs{}}
\PY{l+s+s1}{        }\PY{l+s+s1}{\PYZbs{}}\PY{l+s+s1}{langle L\PYZus{}}\PY{l+s+si}{\PYZob{}}\PY{n}{i}\PY{o}{+}\PY{l+m+mi}{1}\PY{l+s+si}{\PYZcb{}}\PY{l+s+s1}{ }\PY{l+s+s1}{\PYZbs{}}\PY{l+s+s1}{rangle\PYZca{}2 \PYZhy{} }\PY{l+s+s1}{\PYZbs{}}\PY{l+s+s1}{langle L\PYZus{}}\PY{l+s+si}{\PYZob{}}\PY{n}{i}\PY{o}{+}\PY{l+m+mi}{1}\PY{l+s+si}{\PYZcb{}}\PY{l+s+s1}{\PYZca{}2 }\PY{l+s+s1}{\PYZbs{}}\PY{l+s+s1}{rangle = }\PY{l+s+si}{\PYZob{}}\PY{n}{latex}\PY{p}{(}\PY{n}{expr}\PY{p}{)}\PY{l+s+si}{\PYZcb{}}\PY{l+s+s1}{\PYZdl{}}\PY{l+s+s1}{\PYZsq{}}\PY{p}{)}\PY{p}{)}
\end{Verbatim}
\end{tcolorbox}

    \(\displaystyle \
 \langle L_1 \rangle^2 - \langle L_1^2 \rangle = - \frac{I_{1} T \left(-2 + \pi\right)}{\pi}\)

    
    \(\displaystyle \
 \langle L_2 \rangle^2 - \langle L_2^2 \rangle = - \frac{I_{2} T \left(-2 + \pi\right)}{\pi}\)

    
    \(\displaystyle \
 \langle L_3 \rangle^2 - \langle L_3^2 \rangle = - \frac{I_{3} T \left(-2 + \pi\right)}{\pi}\)

    
    \hypertarget{ux437ux430ux434ux430ux43dux438ux435-5}{%
\section{Задание 5}\label{ux437ux430ux434ux430ux43dux438ux435-5}}

    В силу линейности операции усреднения:
\[\langle \omega \rangle = \langle \omega_1 \rangle\]


    % Add a bibliography block to the postdoc
    
    
    
\end{document}
