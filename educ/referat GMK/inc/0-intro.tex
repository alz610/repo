\anonsection{Введение}

С целью получения данных о притоке жидкости в скважину, а в нагнетательных скважинах --- для оценки количества жидкости, поступающей в пласт (приемистость пласта), применяют расходомеры или дебитомеры. Расходомерами измеряют расходы воды, нагнетаемой в скважину, дебитомерами --- притоки нефти, газа и их смеси с водой, поступающей из пласта в скважину \cite{debitraskhod}.

По принципу измерения у данных приборов нет особых различий. Расходомеры отличаются от дебитомеров б\'ольшим диаметром корпуса глубинного прибора \cite{mechdebit}.

Глубинные расходомеры являются важным средством изучения нефтяного месторождения и исследования характера работы нефтяных скважин. С помощью глубинных расходомеров на нефтяных месторождениях решают следующие задачи:
\begin{itemize}
    \item измеряют дебит каждого пласта в отдельности при одновременной раздельной эксплуатации нескольких нефтяных горизонтов одной скважиной;
    \item определяют место и значение притока по вертикали нефтяного горизонта для выявления качества перфорации, эффективности гидравлического разрыва пласта и местообразования трещин;
    \item устанавливают характер притока жидкости из пласта в скважину (изменение притока в зависимости от забойною давления) при гидродинамических исследованиях пласта;
    \item определяют места нарушений герметичности эксплуатационной колонны по изменению притока по стволу скважины; 
    \item устанавливают наличие перетока жидкости из одного продуктивного пропластка в другой \cite{mechdebit}.
\end{itemize}

\clearpage
