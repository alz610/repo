\anonsection{Заключение}

Используют беспакерные и пакерные расходомеры, последние --- только для измерения потоков жидкости. Пакер служит для перекрытия сечения скважины и направления потока жидкости через измерительную камеру, в которую помещена турбинка. При использовании пакера невозможен непрерывный режим записи \cite{mechraskhod}.

На сегодняшний день расходомеры с датчиком турбинного типа более развиты и всё больше находят свое применение в нефтяной и газовой промышленности. То есть альтернативы глубинным механическим расходомерам - как недорогому, достаточно универсальному и хорошо изученному средству измерения объема жидкости и газа в стволе 
эксплуатационной скважины на сегодняшний день не существует \cite{mechdebit}.

Специфические условия, при которых работают глубинные приборы это: большая глубина спуска; высокие температуры и давления; небольшой диаметр скважины; коррозионная активность; удары, тряски и т. д. Для получения высокой точности результатов измерения к чувствительным элементам глубинных приборов выдвигают жесткие требования. К ним относятся: большой диапазон измерений, высокая чувствительность, малая инерционность, работоспособность при больших температурах и давлениях, стабильность во времени, коррозионная стойкость, устойчивость к вибрациям и ударам \cite{mechdebit}. 

\clearpage
